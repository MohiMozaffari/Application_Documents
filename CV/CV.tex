\documentclass[11pt, b4paper]{cv}
\usepackage[top=0.4in, left=0.5in, right=0.5in, bottom=0.4in]{geometry}
\usepackage{fontawesome5}
\usepackage{hyperref}
\usepackage{multicol}
\usepackage{hyphenat}
\tolerance=1
\emergencystretch=\maxdimen
\hyphenpenalty=100000
\hbadness=100000

\renewcommand{\baselinestretch}{1.0} % Adjusted line spacing


\begin{document}

\begin{center}
%	    \textit{R\'esum\'e}\\
	\textit{Curriculum Vitae}\\
	{\Large \textbf{Mohaddeseh Mozaffari}\par}
	%\href{tel:(+98)9012051379}{\faIcon{phone} +98(0) 9012051379}
	\href{mailto:mohaddeseh.mozaffarii@gmail.com}{\faIcon{envelope} mohaddeseh.mozaffarii@gmail.com}
	\hspace{5pt} 
	\href{https://www.linkedin.com/in/MohiMozaffari/}{\faIcon{linkedin} MohiMozaffari}
	\hspace{5pt} 
	\href{https://github.com/MohiMozaffari}{\faIcon{github} MohiMozaffari}
\end{center}


\vspace{-0.15in}
\rule{\textwidth}{1pt}\\
\vspace{-0.15in}

\textbf{EDUCATION}

\begin{education}{Sep 2022}{Present}{Master of Science}{Statistical Physics and Complex Systems}{Shahid Beheshti University}{Tehran, Iran}{18.5 of 20}
\end{education}
\begin{education}{Sep 2018}{May 2022}{Bachelor of Science}{Physics}{Shahid Beheshti University}{Tehran, Iran}{17.3 of 20}
\end{education}

\vspace{-0.15in}
\rule{\textwidth}{1pt}\\
\vspace{-0.15in}

\textbf{RESEARCH INTEREST}
\vspace{-0.1in}

% \begin{itemize}
%     \item \textbf{Complex Systems and Computational Science:}  
%     network dynamics, computational simulations.

%     \item \textbf{Neuroscience and Biological Data Analysis:}  
%     Brain connectivity, cognitive systems, topological data analysis.

%     \item \textbf{Artificial Intelligence and Data Science:}  
%     computer vision, data-driven solutions for medical imaging.
% \end{itemize}

\begin{multicols}{3}
	\begin{itemize}
		\setlength\itemsep{-1pt}
		\item Biological Physics
		\item Cognitive Science
		\item Complex Systems
		\item Computational physics
		\item Data Science
		\item ML \& AI
	\end{itemize}
\end{multicols}

% \begin{multicols}{3}
% 	\begin{itemize}
% 		\item Biological Physics
% 		\item Cognitive Science
% 		\item Network Science
% 		\item Neuroscience
% 		\item Complex Systems
% 		\item Stochastic Processes
% 		\item Computational physics
% 		\item Topological data analysis
% 		\item Data Science
% 		\item Machine Learning
% 		\item Artificial Intelligence
% 		\item Computer Vision
% 	\end{itemize}
% \end{multicols}

\vspace{-0.15in}
\rule{\textwidth}{1pt}\\
\vspace{-0.15in}

\textbf{RESEARCH EXPERIENCE}\\
\vspace{-0.13in}

 \begin{research_exp}{Jan 2024}{Present}{Working on fMRI Data Using Topological Data Analysis (TDA)
 	}{Shahid Beheshti University}{Prof. G.Reza Jafari}
	 
	 \begin{itemize}
	 \item Applied Topological Data Analysis (TDA) to study fMRI data and uncover brain connectivity patterns.
	 \item Analyzed cerebral activity differences between individuals with Autism Spectrum Disorder (ASD) and neurotypical cohorts
	 \item Used age-stratified methods to examine neurodevelopmental trajectories across various age groups.
	\end{itemize}
\end{research_exp}
\vspace{-0.1in}

\begin{research_exp}{Dec 2023}{Present}{Automatic Segmentation of Coronary Artery}{Shahid Rajaei Heart Hospital, Tehran, Iran}{Mehdi Yousefzadeh}
	\begin{itemize}
        \item Collected and processed medical data from the hospital, working with DICOM files to identify and label vessels and catheters.
        \item Developed custom augmentation techniques to enhance data quality and improve segmentation accuracy.	\end{itemize}
	
\end{research_exp}
\vspace{-0.1in}

\begin{research_exp}{Aug 2022}{Oct 2022}{Abnormality Detection in Head CT Scan Reports using Text Mining
	}{Imam Hossein Hospital Research Council, Tehran, Iran}{Mehdi Yousefzadeh}
	
	\begin{itemize}
		\item Conducted text mining and preprocessing on radiology reports to detect abnormalities.
	\end{itemize}
\end{research_exp}
    

\vspace{-0.15in}
\rule{\textwidth}{1pt}\\
\vspace{-0.15in}

\textbf{PUBLICATIONS}
\vspace{-0.10in}

\begin{itemize}
	\item 	Mohammadi, M.S., Shahrokhi, S., \textbf{Mozaffari, M.} et al. Nonlinear optical response of IMIP ionic
	liquid-stabilized magnetic graphene oxide sheets. J Mater Sci: Mater Electron 33, 13224–13233
	(2022).
\end{itemize}


\vspace{-0.15in}
\rule{\textwidth}{1pt}\\
\vspace{-0.15in}

\textbf{HONORS AND AWARDS}
\vspace{-0.10in}

\begin{itemize}
    \item \textbf{Second rank} in M.Sc. Statistical Physics and Complex Systems, Shahid Beheshti University (2023).
	\item Admitted to the master’s program through \textbf{the Exceptional Talent initiative} for top students (2022).
	\item \textbf{First rank} among B.Sc. students, Physics Department, Shahid Beheshti University (2021-2022).
    \item \textbf{Top 1\%} in nationwide university entrance exam in Iran (2018).
\end{itemize}


\vspace{-0.15in}
\rule{\textwidth}{1pt}\\
\vspace{-0.15in}

\textbf{SKILLS}

% \begin{itemize}
%     \item \textbf{Programming Languages:} Python (Advanced), C++ (Intermediate), Bash/Linux (Intermediate), HTML (Elementary), CSS (Elementary)
%     \item \textbf{Software \& Tools:} Git (Advanced), \LaTeX{} (Advanced), Microsoft Office Suite (Advanced), Adobe Illustrator (Advanced), Adobe Photoshop (Intermediate)
%     \item \textbf{Other Skills:} Data Analysis, Machine Learning, Numerical Simulations
% \end{itemize}

\textit{Cumputing}
\vspace{-0.1in}

\begin{multicols}{3}[]
	\begin{itemize}
	\setlength\itemsep{-0.5pt}
	  \item  Python (Advanced)
	  \item  C++ (Intermediate)
	  \item  Git (Intermediate)
	  \item  Bash/Linux (Intermediate)
	  \item  Adobe Illustrator (Advanced)
	  \item  Adobe Photoshop (Intermediate)
	  \item  HTML, CSS (Elementary)
	  \item  \LaTeX  (Advanced)
	  \item  Microsoft Office Suite: Word, Excel, PowerPoint (Advanced)
	\end{itemize}
\end{multicols}

\vspace{-0.1in}
\textit{Languages}
\vspace{-0.1in}

\begin{multicols}{3}[\columnsep=0cm]
	\begin{itemize}
	\setlength\itemsep{-1pt}
	  \item  Persian (Native)
	  \item  English (Fluent)
	\end{itemize}
\end{multicols}

\pagebreak
\rule{\textwidth}{1pt}\\
\vspace{-0.15in}

\textbf{WORK EXPERIENCE}

\begin{work}{Jul 2024}{Present}{Python Instructor}{Ostadbank, Tehran, Iran}
	\begin{itemize}
		\item Teaching Python object-oriented programming, machine learning, and AI to students, focusing on real-world applications and mini projects.
	\end{itemize}
	%Ostadbak is a platform connecting private tutors with learners. I teach Python object-oriented programming, machine learning, and AI, helping students build mini projects to apply their skills in real-world scenarios.
\end{work}

\vspace{-0.1in}

\begin{work}{Jun 2023}{Present}{Python Instructor}{Picha Club, Tehran, Iran}
	\begin{itemize}
		\item Instructing pre-teens and teens in Python fundamentals, algorithms, and Tkinter.
		\item Guiding students through game development projects.
	\end{itemize}
	%Picha is an online programming school for pre-teens and teens. I teach Python fundamentals, algorithms, object-oriented programming, and Tkinter for building applications, while also guiding students through game development using classes and projects.
\end{work}

\vspace{-0.15in}
\rule{\textwidth}{1pt}
\vspace{-0.15in}

\textbf{TEACHING EXPERIENCE}

\textbf{Position:} Teaching Assistant\\
\textbf{Where:} \hspace{0.1in}Department of Physics, Shahid Beheshti University, Tehran, Iran\\
\textbf{Courses:}
\begin{itemize}
	\item Stochastic Processes (Jan 2024 – Jul 2024)
	\item Foundations of Numerical Simulations (Sep 2023 – Jan 2024)
	\item Complex Systems Physics (Sep 2023 – Jan 2024)
	\item Analytical Mechanics (Sep 2022 – Jan 2023)
\end{itemize}    

% \textbf{Key Responsibilities:}

% \begin{adjustwidth}{}{\rightedge}
% \begin{itemize}
%     \item Designed problem sets and solution classes to reinforce theoretical concepts.
%     \item Taught Python programming and numerical simulations to students, helping them apply computational techniques to solve complex physics problems.
%     \item Guided students in hands-on projects, focusing on the practical application of course materials.
%     \item Provided one-on-one support during office hours to clarify course content and assist with assignments.
% \end{itemize}
% \end{adjustwidth}

\vspace{-0.15in}
\rule{\textwidth}{1pt}\\
\vspace{-0.15in}

\textbf{CERTIFICATIONS}
\vspace{-0.1in}

\begin{itemize}
	\item Machine Learning Specialization, Coursera (2023).
	\item Neural Networks and Deep Learning, Coursera (2022).
	\item Quantum Computing and Implementation in Python, Qorpi Engineering Workgroup – KNTU Innovation Center (2022).
	\item Quantum Information and Computers, Interdisciplinary Schools, Sharif University of Technology (2021).
\end{itemize}

\vspace{-0.15in}
\rule{\textwidth}{1pt}\\
\vspace{-0.15in}

\textbf{CONFERENCES \& SCHOOLS ATTENDED}

\begin{school}{Sep 2024}{Sep 2024}{Shahrood University of Technology, Semnan, Iran}{5th National Conference on Quantum Information and Computing}
	\begin{itemize}
	    \item Attended presentations on the latest advancements in quantum information theory and quantum computing.
	\end{itemize}
\end{school}
\vspace{-0.1in}

\begin{school}{Oct 2023}{Dec 2023}{School of Biological Sciences, IPM, Tehran, Iran}{The School of Evolutionary Dynamics of Cells and Viruses}
	\begin{itemize}
	    \item Participated in lectures and discussions on evolutionary dynamics in cells and viruses.
	    \item Explored theoretical models and their biological applications.
	\end{itemize}
\end{school}
\vspace{-0.1in}

\begin{school}{Jul 2023}{Jul 2023}{Institute for Advanced Studies in Basic Science, Zanjan, Iran}{The $28^{th}$ Special School on Topics in Physics}
	\begin{itemize}
	    \item Engaged in advanced discussions on topics like condensed matter physics, quantum mechanics, and biosensing technologies.
	\end{itemize}
\end{school}
% \vspace{-0.15in}
% \rule{\textwidth}{1pt}\\
% \vspace{-0.1in}


% \textbf{LANGUAGES}

% \begin{multicols}{3}[]
% 	\begin{itemize}
% 	  \item  Persian (Native)
% 	  \item  English (Fluent)
% 	\end{itemize}
% \end{multicols}
\vspace{-0.15in}
\rule{\textwidth}{1pt}\\
\vspace{-0.15in}

\textbf{PERSONAL INFORMATION}
\vspace{-0.1in}

\begin{multicols}{3}[\columnsep=0cm]
	\begin{itemize}
	  \item Gender: Female
	  \item Nationality: Iranian
	  \item Date of Birth: Aug. $10^{th}$, 2000
	  %\item Hobbies: Walking, Reading Books, Painting
	\end{itemize}
\end{multicols}

\vspace{-0.15in}
\rule{\textwidth}{1pt}\\
\vspace{-0.15in}

\textbf{REFERENCES}
\vspace{-0.1in}

\begin{itemize}
	\item \textbf{Gholamreza Jafari}, Professor of Physics, Department of Physics and Institute for Cognitive Science and Brian, Shahid Beheshti University, Tehran, Iran.
	
	\href{tel:(+98)2129902773}{\faIcon{phone} (+98) 21 2990 2773}\\
	%\hspace{1in}
	\href{mailto:gjafari@gmail.com}{\faIcon{envelope} gjafari@gmail.com}\\
	\href{https://complexity.sbu.ac.ir/professor-reza-jafari/}{\faIcon{globe}HomePage} 
	\item \textbf{Ali Hosseiny Esfidvajani}, Assistant Professor, Faculty of Physics, Shahid Beheshti University, Tehran, Iran.
	
	\href{tel:(+98)2129905043}{\faIcon{phone} (+98) 21 2990 5043}\\
	%\hspace{1in}
	\href{mailto:alihd22@gmail.com}{\faIcon{envelope} alihd22@gmail.com}\\
	\href{https://alihosseiny.com/}{\faIcon{globe}HomePage} 
		
	\item \textbf{Marzieh Farhang}, Associate Professor, Faculty of Physics, Shahid Beheshti University, Tehran, Iran.
	
	\href{tel:(+98)2129905053}{\faIcon{phone} (+98) 21 2990 5053}\\
	%\hspace{1in}
	\href{mailto:marzieh.farhang@gmail.com}{\faIcon{envelope} marzieh.farhang@gmail.com}\\
	\href{https://en.sbu.ac.ir/~m_farhang/home}{\faIcon{globe}HomePage} 
\end{itemize}
\end{document}
