\documentclass[12pt, b4paper]{cv}
\usepackage[top=0.5in, left=0.5in, right=0.5in, bottom=0.5in]{geometry}
\usepackage{fontawesome5}
\usepackage{multicol}
\usepackage{hyphenat}
\usepackage{xcolor}

\definecolor{purple}{RGB}{100,50,150}

\usepackage[
    colorlinks = true,
    linkcolor  = purple,
    urlcolor   = purple,
    citecolor  = purple
]{hyperref}

\tolerance=1
\emergencystretch=\maxdimen
\hyphenpenalty=100000
\hbadness=100000

\renewcommand{\baselinestretch}{1.} % Adjusted line spacing


\begin{document}

\begin{center}
%	    \textit{R\'esum\'e}\\
	\textit{Curriculum Vitae}\\
	{\Huge \textbf{Mohaddeseh Mozaffari}\par}

	%\href{tel:(+98)9012051379}{\faIcon{phone} +98(0) 9012051379}
	\href{mailto:mohaddeseh.mozaffarii@gmail.com}{\faIcon{envelope} mohaddeseh.mozaffarii@gmail.com}
	\hspace{5pt} 
	\href{https://www.linkedin.com/in/MohiMozaffari/}{\faIcon{linkedin} MohiMozaffari}
	\hspace{5pt} 
	\href{https://github.com/MohiMozaffari}{\faIcon{github} MohiMozaffari}
	\hspace{5pt} 
	\href{https://mohimozaffari.github.io/}{\faIcon{globe} Personal Website} 
\end{center}


\vspace{-0.15in}
\rule{\textwidth}{1pt}\\
\vspace{-0.15in}

%%%%%%%%%%%%%%

% education
{\Large \textbf{EDUCATION}}
\vspace{0.1in}


\begin{education}{Sep 2022 -- Mar 2025}{M.Sc.}{Statistical Physics and Complex Systems}{Shahid Beheshti University}{Tehran, Iran}{{18.5/20 (\textasciitilde 4.00/4.00 WES, \textbf{Second Rank})}}%; Admitted via Exceptional Talent Initiative}
\end{education}

\begin{education}{Sep 2018 -- May 2022}{B.Sc.}{Physics}{Shahid Beheshti University}{Tehran, Iran}{17.3/20 (\textasciitilde 3.63/4.00 WES, \textbf{First Rank})}
\end{education}

\vspace{-0.15in}
\rule{\textwidth}{1pt}\\
\vspace{-0.15in}


%%%%%%%%%%%%%%%%%
{\Large \textbf{RESEARCH INTERESTS}}

\vspace{-5pt}
\begin{multicols}{3}
	\begin{itemize}
		\setlength\itemsep{-1pt}
				\item Bio-Physics
				\item Computional Neuroscience
				\item Brain Network Analysis
				\item Network Neuroscience
				\item Machine Learning
				\item Artificial Intelligence
	\end{itemize}
\end{multicols}



\vspace{-0.15in}
\rule{\textwidth}{1pt}\\
\vspace{-0.15in}

%%%%%%%%%%%%%%%%%%%%%%
{\Large \textbf{RESEARCH EXPERIENCE}}
\vspace{0.1in}

\begin{research_exp}{Jul 2025 -- Present}{\textbf{Coevolutionary and Structural Balance Network Analysis and Classification of ADHD Using the Open-Source ADHD-400 Dataset}\vspace{5pt}}{Center for Complex Networks (CCNet), Tehran, Iran}{\href{https://complexity.sbu.ac.ir/professor-reza-jafari/}{Prof. Reza Jafari}}

\vspace{10pt}
\begin{itemize}
    \item Collaborated on the application of stuctural and Coevolutionary balance theory on brain networks.

    \item Engineered balance-theoretic features for group-level differentiation.

    \item Trained machine learning models to classify ADHD vs. control subjects.

    \item Contributed to drafting, editing, and reviewing the manuscript for publication.
\end{itemize}
\end{research_exp}


\begin{research_exp}{Jan 2024 -- Present}{\textbf{Master’s Thesis – Analysis of Topological Features of Brain Networks in the Autism Spectrum Disorder and Control Group Using Persistent Homology}\vspace{5pt}}{Shahid Beheshti University, Tehran, Iran}{\href{https://complexity.sbu.ac.ir/professor-reza-jafari/}{Prof. Reza Jafari}}

\vspace{10pt}
\begin{itemize}
	
	\item Applied topological data analysis (TDA) and persistent homology on fMRI data.

	\item Developed a node-removal-based approach to detect differences in topological features.
	
	\item Investigated age-related differences in brain network topology.

	\item Trained machine learning models to classify study groups.

	\item Developed a private Python package, \texttt{NeuroPHorm}, to automate the full TDA workflow.
\end{itemize}
\end{research_exp}


% \begin{research_exp}{Dec 2023 -- Present}{\textbf{Automatic Classification and Segmentation of Coronary Arteries Using AI}}{Shahid Rajaei Heart Hospital, Tehran, Iran}{Dr. Mehdi Yousefzadeh}
% \vspace{10pt}
% 	\begin{itemize}
% 		\item Collaborated on building an angiography dataset from X-ray Coronary Angiography (XCA), including DICOM handling, data cleaning, and vessel annotation.

% 		\item Co-developed segmentation pipelines using classical image processing (Frangi, Sato, Meijering filters) and deep learning models (U-Net, U-Net3+, FPN).

% 		\item Contributed specifically to image enhancement using the Meijering and Sato filters for coronary vessel segmentation.

% 		\item Achieved a 93\% Dice score on the test set, exceeding performance of many recent segmentation approaches.
% 	\end{itemize}
	
% \end{research_exp}


\vspace{-0.15in}
\rule{\textwidth}{1pt}\\
\vspace{-0.15in}

{\Large \textbf{PUBLICATIONS}}
\vspace{5pt}

% ---------- Journal Articles ----------
\textit{Journal Articles}
\begin{itemize}
\item Mohammadi, M.S., Shahrokhi, S., \textbf{Mozaffari, M.} \textit{et al.}
    Nonlinear optical response of IMIP ionic liquid-stabilized magnetic graphene oxide sheets.
    \textit{Journal of Materials Science: Materials in Electronics}, 33, 13224–13233 (2022).
    \href{https://doi.org/10.1007/s10854-022-08262-1}{DOI:10.1007/s10854-022-08262-1}

\end{itemize}

% ---------- Conference Papers ----------
\textit{Conference Papers}
\begin{itemize}
    \item Yousefzadeh, M., Shirzadeh Barough, S., Fakharifar, A., \textbf{Mozaffari, M.}, \textit{et al.}
    Automated Noninvasive FFR Estimation from Biplane Coronary Angiography Using a Transformer-Based Deep Learning Framework.
    \textit{The Second National Meeting on Artificial Intelligence in Medical Imaging} (Oral Presentation), Rajaee Heart Institute, Tehran, Iran, June 11–13, 2025.
\end{itemize}

% ---------- Manuscripts in Preparation ----------
\textit{Manuscripts in Preparation}
\begin{itemize}
    \item \textbf{Mozaffari, M.}, Roshandel, S., Jafari, G.R.
    Persistent Homology Reveals Topological Alterations in Resting-State Brain Networks of Autism Spectrum Disorder.

    \item Yousefzadeh, M., Shirzadeh Barough, S., Fakharifar, A., Tayyarazad, Y., Eghbali, N., \textbf{Mozaffari, M.}, \textit{et al.}
    Coronary Artery Segmentation and Vessel-Type Classification in X-Ray Angiography: Machine-Learning Generalized Image Processing and Deep Neural Networks.

\end{itemize}


\pagebreak

% \vspace{-0.15in}
% \rule{\textwidth}{1pt}\\
% \vspace{-0.15in}

% \textbf{HONORS AND AWARDS}
% \vspace{-0.10in}

% \begin{itemize}
%     % \item \textbf{Second rank} in M.Sc. Statistical Physics and Complex Systems, Shahid Beheshti University (2023).
% 	\item Admitted to the master’s program through \textbf{the Exceptional Talent initiative} for top students (2022).
% 	% \item \textbf{First rank} among B.Sc. students, Physics Department, Shahid Beheshti University (2021-2022).
%     \item \textbf{Top 1\%} in nationwide university entrance exam in Iran (2018).
% \end{itemize}

\vspace{-0.15in}
\rule{\textwidth}{1pt}\\
\vspace{-0.15in}

% Skills

{\Large \textbf{SKILLS}}
\vspace{5pt}

\textit{Computing}

\begin{multicols}{3}[]
	\begin{itemize}
	\setlength\itemsep{-0.7pt}
	  \item  Python (Advanced)
	  \item  C++ (Intermediate)
	  \item  Git (Intermediate)
	  \item  Bash/Linux (Intermediate)
	  \item  Adobe Illustrator (Advanced)
	  \item  Adobe Photoshop (Intermediate)
	  \item  HTML/CSS (Elementary)
	  \item  \LaTeX  (Advanced)
	  \item  Microsoft Office Suite: Word, Excel, PowerPoint (Advanced)
	\end{itemize}
\end{multicols}

\textit{Languages}

\begin{multicols}{3}[]
	\begin{itemize}
	\setlength\itemsep{-1pt}
	  \item  Persian (Native)
	  \item  English (Fluent)
	\end{itemize}
\end{multicols}

%\pagebreak
\vspace{-0.15in}
\rule{\textwidth}{1pt}\\
\vspace{-0.15in}

{\Large \textbf{TEACHING EXPERIENCE}}
\vspace{0.1in}

\begin{work}{}{Teaching Assistant}{Department of Physics, Shahid Beheshti University}
	\vspace{-0.2in}
	\begin{multicols}{2}
		\begin{itemize}
			\item Complex Systems Physics \textit{(Jan 2025 – Jul 2025)}
			\item Complex Networks and Graph Theory \textit{(Jan 2025 – Jul 2025)}
			\item Stochastic Processes \textit{(Jan 2024 – Jul 2024)}
			\item Foundations of Numerical Simulations \textit{(Sep 2023 – Jan 2024)}
			\item Complex Systems Physics \textit{(Sep 2023 – Jan 2024)}
			\item Analytical Mechanics \textit{(Sep 2022 – Jan 2023)}
		\end{itemize}    
	\end{multicols}
\end{work}
\vspace{-0.15in}



\vspace{-0.15in}
\rule{\textwidth}{1pt}\\
\vspace{-0.15in}

{\Large \textbf{WORK EXPERIENCE}}
\vspace{0.1in}

\begin{work}{Jul 2024 -- Present}{Python Instructor}{Ostadbank, Tehran, Iran}
\vspace{-0.05in}
\begin{itemize}
    \item Deliver tailored Python lessons on OOP, ML, and AI to diverse learners.
    \item Guide students in mini-projects using NumPy, pandas, Matplotlib, seaborn, scikit-learn, and PyTorch.
    % \item \href{https://www.ostadbank.com/tutor/mohimozaffari}{Public profile (Persian)}
\end{itemize}
\end{work}

\vspace{-0.05in}

\begin{work}{Jun 2023 -- Present}{Python Instructor}{Picha Club, Tehran, Iran}
\vspace{-0.05in}
\begin{itemize}
    \item Teach Python fundamentals, algorithms, and OOP to pre-teens and teens.
    \item Support students in building Tkinter apps and Pygame games.
    % \item \href{https://pichaclub.ir/m-mozafari/}{Public profile (Persian)}
\end{itemize}
\end{work}




\vspace{-0.15in}
\rule{\textwidth}{1pt}
\vspace{-0.15in}



{\Large \textbf{INVITED TALKS}}
\vspace{0.1in}

\begin{school}{Apr 2025}{Yasouj University, Yasouj, Iran}{Statistical Physics and Complex Systems}
\vspace{-0.1in}
\begin{itemize}
    \item Introduced undergraduate physics students to complex systems in an invited online Persian talk
          \href{https://www.aparat.com/v/szyh283}{(Recording available)}.
    % \item Outlined research and career pathways in statistical physics and complex systems.
\end{itemize}
\end{school}



\vspace{-0.15in}
\rule{\textwidth}{1pt}\\
\vspace{-0.15in}

{\Large \textbf{CERTIFICATIONS}}
\vspace{-5pt}

\begin{itemize}
    \item Deep Learning (Python) for Neuroscience EEG Practical Course — Udemy, Instructor: Ildar Rakhmatulin \textit{(Aug 2025)}
          \href{https://www.udemy.com/certificate/UC-1c61cc70-1c1f-49fc-b6b1-972fa68f97df/}{(Certificate)}
    \item Machine Learning Specialization — Coursera / Stanford Online, Instructor: Andrew Ng \textit{(Sep 2023)}
          \href{https://coursera.org/share/73ec39657746dc6a319fa5a123047ccf}{(Certificate)}
    \item Neural Networks and Deep Learning — DeepLearning.AI / Coursera, Instructor: Andrew Ng \textit{(Aug 2022)}
          \href{https://coursera.org/share/601e36dd06712caa01ff9b867a6bf7a7}{(Certificate)}
\end{itemize}



\vspace{-0.15in}
\rule{\textwidth}{1pt}\\
\vspace{-0.15in}

{\Large \textbf{WORKSHOPS, SCHOOLS, AND CONFERENCES ATTENDED}}
\vspace{-5pt}

\begin{itemize}
    \item fMRI Image Processing With CONN Toolbox — Shahid Beheshti University, Tehran, Iran \textit{(Nov 2024)}
    % \item 5th National Conference on Quantum Information and Computing — Shahrood University of Technology, Semnan, Iran (Sep 2024)
    \item The School of Evolutionary Dynamics of Cells and Viruses — School of Biological Sciences, IPM, Tehran, Iran \textit{(Dec 2023)}
    \item The $28^{th}$ Special School on Topics in Physics — Institute for Advanced Studies in Basic Science, Zanjan, Iran \textit{(Jul 2023)}
\end{itemize}
% \begin{school}{Nov 2024}{Shahid Beheshti University, Tehran, Iran}{fMRI Image Processing With CONN Toolbox}
% \vspace{-0.1in}
% 	\begin{itemize}
% 	    \item Gained hands-on experience in preprocessing, denoising, and connectivity analysis for resting-state and task-based fMRI using the CONN toolbox.
% 	\end{itemize}
% \end{school}


% \begin{school}{Sep 2024}{Shahrood University of Technology, Semnan, Iran}{5th National Conference on Quantum Information and Computing}
% \vspace{-0.1in}
% 	\begin{itemize}
% 	    \item Attended presentations on the latest advancements in quantum information theory and quantum computing.
% 	\end{itemize}
% \end{school}


% \begin{school}{Dec 2023}{School of Biological Sciences, IPM, Tehran, Iran}{The School of Evolutionary Dynamics of Cells and Viruses}
% \vspace{-0.1in}
% 	\begin{itemize}
% 	    \item Participated in lectures and discussions on evolutionary dynamics in cells and viruses.
% 	    \item Explored theoretical models and their biological applications.
% 	\end{itemize}
% \end{school}


% \begin{school}{Jul 2023}{Institute for Advanced Studies in Basic Science, Zanjan, Iran}{The $28^{th}$ Special School on Topics in Physics}
% \vspace{-0.1in}
% 	\begin{itemize}
% 	    \item Engaged in advanced discussions on topics like condensed matter physics, quantum mechanics, and biosensing technologies.
% 	\end{itemize}
% \end{school}




\vspace{-0.15in}
\rule{\textwidth}{1pt}\\
\vspace{-0.15in}




% \textbf{PERSONAL INFORMATION}
% \vspace{-0.1in}

% \begin{multicols}{3}[\columnsep=0cm]
% 	\begin{itemize}
% 	  \item Gender: Female
% 	  \item Nationality: Iranian
% 	  \item Date of Birth: Aug. $10^{th}$, 2000
% 	  %\item Hobbies: Walking, Reading Books, Painting
% 	\end{itemize}
% \end{multicols}

% \vspace{-0.15in}
% \rule{\textwidth}{1pt}\\
% \vspace{-0.15in}

{\Large \textbf{REFERENCES}}
\vspace{-5pt}

\begin{itemize}
	\item \textbf{Reza Jafari}, Professor of Physics, Department of Physics and Institute for Cognitive Science and Brian, Shahid Beheshti University, Tehran, Iran.\\
	\href{tel:(+98)2129902773}{\faIcon{phone} (+98) 21 2990 2773}
	\hspace{0.5in}
	\href{mailto:g\_jafari@sbu.ac.ir}{\faIcon{envelope} g\_jafari@sbu.ac.ir}
	\hspace{0.75in}
	\href{mailto:gjafari@gmail.com}{\faIcon{envelope} gjafari@gmail.com}
	\hspace{1.14in}
	\href{https://complexity.sbu.ac.ir/professor-reza-jafari/}{\faIcon{globe}HomePage} 
	\item \textbf{S. Ali Hosseiny Esfidvajani}, Assistant Professor, Faculty of Physics, Shahid Beheshti University, Tehran, Iran.\\
	\href{tel:(+98)2129905043}{\faIcon{phone} (+98) 21 2990 5043}
	\hspace{0.5in}
	\href{mailto:al\_hosseiny@sbu.ac.ir}{\faIcon{envelope} al\_hosseiny@sbu.ac.ir}
	\hspace{0.5in}
	\href{mailto:alihd22@gmail.com}{\faIcon{envelope} alihd22@gmail.com}
	\hspace{1.091in}
	\href{https://alihosseiny.com/}{\faIcon{globe}HomePage} 
	\item \textbf{Marzieh Farhang}, Associate Professor, Faculty of Physics, Shahid Beheshti University, Tehran, Iran.\\
	\href{tel:(+98)2129905053}{\faIcon{phone} (+98) 21 2990 5053}
	\hspace{0.5in}
	\href{mailto:m\_farhang@sbu.ac.ir}{\faIcon{envelope} m\_farhang@sbu.ac.ir}
	\hspace{0.54in}
	\href{mailto:marzieh.farhang@gmail.com}{\faIcon{envelope} marzieh.farhang@gmail.com}
	\hspace{0.5in}
	\href{https://en.sbu.ac.ir/~m_farhang/home}{\faIcon{globe}HomePage} 
\end{itemize}
\end{document}
