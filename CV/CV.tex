\documentclass[12pt, b4paper]{cv}
\usepackage[top=0.4in, left=0.5in, right=0.5in, bottom=0.4in]{geometry}
\usepackage{fontawesome5}
\usepackage{hyperref}
\usepackage{multicol}
\usepackage{hyphenat}


\tolerance=1
\emergencystretch=\maxdimen
\hyphenpenalty=100000
\hbadness=100000

\renewcommand{\baselinestretch}{1.0} % Adjusted line spacing


\begin{document}

\begin{center}
%	    \textit{R\'esum\'e}\\
	\textit{Curriculum Vitae}\\
	{\Huge \textbf{Mohaddeseh Mozaffari}\par}

	%\href{tel:(+98)9012051379}{\faIcon{phone} +98(0) 9012051379}
	\href{mailto:mohaddeseh.mozaffarii@gmail.com}{\faIcon{envelope} mohaddeseh.mozaffarii@gmail.com}
	\hspace{5pt} 
	\href{https://www.linkedin.com/in/MohiMozaffari/}{\faIcon{linkedin} MohiMozaffari}
	\hspace{5pt} 
	\href{https://github.com/MohiMozaffari}{\faIcon{github} MohiMozaffari}
	\hspace{5pt} 
	\href{https://mohimozaffari.github.io/}{\faIcon{globe} Personal Website} 
\end{center}


\vspace{-0.15in}
\rule{\textwidth}{1pt}\\
\vspace{-0.15in}

%%%%%%%%%%%%%%

% education
{\Large \textbf{EDUCATION}}
\vspace{0.1in}

\begin{education}{Sep 2022 -- Mar 2025}{Master of Science}{Statistical Physics and Complex Systems}{Shahid Beheshti University}{Tehran, Iran}{18.5/20 (\textbf{Second} Rank)}%; Admitted via Exceptional Talent Initiative}
\end{education}

\begin{education}{Sep 2018 -- May 2022}{Bachelor of Science}{Physics}{Shahid Beheshti University}{Tehran, Iran}{17.3/20 (\textbf{First} Rank)}
\end{education}


\vspace{-0.15in}
\rule{\textwidth}{1pt}\\
\vspace{-0.15in}


%%%%%%%%%%%%%%%%%
{\Large \textbf{RESEARCH INTERESTS}}

\vspace{-5pt}
\begin{multicols}{3}
	\begin{itemize}
		\setlength\itemsep{-1pt}
				\item Bio-Physics
				\item Computional Neuroscience
				\item Brain Network Analysis
				\item Network Neuroscience
				\item Machine Learning
				\item Artificial Intelligence
	\end{itemize}
\end{multicols}



\vspace{-0.15in}
\rule{\textwidth}{1pt}\\
\vspace{-0.15in}

%%%%%%%%%%%%%%%%%%%%%%
{\Large \textbf{RESEARCH EXPERIENCE}}
\vspace{0.1in}

\begin{research_exp}{Jul 2025 -- Present}{\textbf{Coevolutionary and Structural Balance Network Analysis and Classification of ADHD Using the Open-Source ADHD-400 Dataset}\vspace{5pt}}{Center for Complex Networks (CCNet), Tehran, Iran}{Prof. Reza Jafari}

\vspace{10pt}
\begin{itemize}
    \item Collaborated on the development and implementation of Coevolutionary and Structural Balance Theories to extract motif-based energy and imbalance metrics from functional brain networks.

    \item Engineered balance-theoretic features from motif structures, network energy profiles, and polarity patterns for group-level differentiation.

    \item Designed and trained machine learning models to classify ADHD vs. control subjects based on extracted topological features.

    \item Contributed to drafting, editing, and reviewing the manuscript for publication.
\end{itemize}
\end{research_exp}


\begin{research_exp}{Jan 2024 -- Present}{\textbf{Master’s Thesis – Analysis of Topological Features of Brain Networks in the Autism Spectrum Disorder and Control Group Using Persistent Homology}\vspace{5pt}}{Shahid Beheshti University, Tehran, Iran}{Prof. Reza Jafari}

\vspace{10pt}
\begin{itemize}
	\item Applied topological data analysis (TDA) and persistent homology to fMRI data, utilizing Vietoris–Rips and Sparse Rips filtrations to identify topological differences in ASD brain networks.

	\item Developed a node-removal-based approach to detect significant changes in the frontoparietal subnetwork of ASD subjects, using Bottleneck and Wasserstein distances for quantification.
	
	\item Investigated age-related differences in brain network topology (childhood, adolescence, adulthood), highlighting connected components and loops as key indicators of ASD.
	
	\item Trained machine learning models using topological features to classify ASD vs. control subjects and predict age groups, demonstrating the potential for enhanced diagnostic accuracy.
	
	\item Developed a private Python package, \texttt{NeuroPHorm}, to streamline and automate the full TDA workflow; currently under internal use and documentation for potential release.
\end{itemize}
\end{research_exp}


\begin{research_exp}{Dec 2023 -- Present}{\textbf{Automatic Classification and Segmentation of Coronary Arteries Using AI}}{Shahid Rajaei Heart Hospital, Tehran, Iran}{Dr. Mehdi Yousefzadeh}
\vspace{10pt}
	\begin{itemize}
		\item Collaborated on building an angiography dataset from X-ray Coronary Angiography (XCA), including DICOM handling, data cleaning, and vessel annotation.

		\item Co-developed segmentation pipelines using classical image processing (Frangi, Sato, Meijering filters) and deep learning models (U-Net, U-Net3+, FPN).

		\item Contributed specifically to image enhancement using the Meijering and Sato filters for coronary vessel segmentation.

		\item Achieved a 93\% Dice score on the test set, exceeding performance of many recent segmentation approaches.
	\end{itemize}
	
\end{research_exp}


\vspace{-0.15in}
\rule{\textwidth}{1pt}\\
\vspace{-0.15in}

{\Large \textbf{PUBLICATIONS}}
\vspace{-5pt}

\begin{itemize}
	\item Mohammadi, M.S., Shahrokhi, S., \textbf{Mozaffari, M.} et al. Nonlinear optical response of IMIP ionic liquid-stabilized magnetic graphene oxide sheets. J Mater Sci: Mater Electron 33, 13224–13233 (2022).
\end{itemize}

\pagebreak

% \vspace{-0.15in}
% \rule{\textwidth}{1pt}\\
% \vspace{-0.15in}

% \textbf{HONORS AND AWARDS}
% \vspace{-0.10in}

% \begin{itemize}
%     % \item \textbf{Second rank} in M.Sc. Statistical Physics and Complex Systems, Shahid Beheshti University (2023).
% 	\item Admitted to the master’s program through \textbf{the Exceptional Talent initiative} for top students (2022).
% 	% \item \textbf{First rank} among B.Sc. students, Physics Department, Shahid Beheshti University (2021-2022).
%     \item \textbf{Top 1\%} in nationwide university entrance exam in Iran (2018).
% \end{itemize}

\vspace{-0.15in}
\rule{\textwidth}{1pt}\\
\vspace{-0.15in}

% Skills

{\Large \textbf{SKILLS}}
\vspace{5pt}

\textit{Computing}

\begin{multicols}{3}[]
	\begin{itemize}
	\setlength\itemsep{-0.5pt}
	  \item  Python (Advanced)
	  \item  C++ (Intermediate)
	  \item  Git (Intermediate)
	  \item  Bash/Linux (Intermediate)
	  \item  Adobe Illustrator (Advanced)
	  \item  Adobe Photoshop (Intermediate)
	  \item  HTML/CSS (Elementary)
	  \item  \LaTeX  (Advanced)
	  \item  Microsoft Office Suite: Word, Excel, PowerPoint (Advanced)
	\end{itemize}
\end{multicols}

\textit{Languages}

\begin{multicols}{3}[]
	\begin{itemize}
	\setlength\itemsep{-1pt}
	  \item  Persian (Native)
	  \item  English (Fluent)
	\end{itemize}
\end{multicols}

%\pagebreak
\vspace{-0.15in}
\rule{\textwidth}{1pt}\\
\vspace{-0.15in}

{\Large \textbf{TEACHING EXPERIENCE}}
\vspace{0.1in}

\begin{work}{}{Teaching Assistant}{Department of Physics, Shahid Beheshti University}
	\vspace{-0.3in}
	\begin{multicols}{2}
		\begin{itemize}
			\item Complex Systems Physics (Jan 2025 – Jul 2025)
			\item Complex Networks and Graph Theory (Jan 2025 – Jul 2025)
			\item Stochastic Processes (Jan 2024 – Jul 2024)
			\item Foundations of Numerical Simulations (Sep 2023 – Jan 2024)
			\item Complex Systems Physics (Sep 2023 – Jan 2024)
			\item Analytical Mechanics (Sep 2022 – Jan 2023)
		\end{itemize}    
	\end{multicols}
\end{work}
\vspace{-0.15in}



\vspace{-0.15in}
\rule{\textwidth}{1pt}\\
\vspace{-0.15in}

{\Large \textbf{WORK EXPERIENCE}}
\vspace{0.1in}

\begin{work}{Jul 2024 -- Present}{Python Instructor}{Ostadbank, Tehran, Iran}
\vspace{-0.1in}
	\begin{itemize}
		\item Deliver tailored Python lessons on OOP, ML, and AI to diverse learners.
		\item Guide students in mini-projects using scikit-learn, pandas, Matplotlib, Keras, and TensorFlow.
	\end{itemize}

\end{work}

\vspace{-0.1in}

\begin{work}{Jun 2023 -- Present}{Python Instructor}{Picha Club, Tehran, Iran}
\vspace{-0.1in}
\begin{itemize}
		\item Teach Python fundamentals, algorithms, and OOP to pre-teens and teens.
		\item Support students in building Tkinter apps and Pygame games.
	\end{itemize}
\end{work}




\vspace{-0.15in}
\rule{\textwidth}{1pt}
\vspace{-0.15in}



{\Large \textbf{INVITED TALKS}}
\vspace{0.1in}

\begin{school}{Apr 2025}{Yasouj University, Yasouj, Iran}{Statistical Physics and Complex Systems}
\vspace{-0.1in}
\begin{itemize}
	\item Introduced undergraduate physics students to complex systems in an invited online Persian talk.
	% \item Outlined research and career pathways in statistical physics and complex systems.
\end{itemize}
\end{school}



\vspace{-0.15in}
\rule{\textwidth}{1pt}\\
\vspace{-0.15in}

{\Large \textbf{CERTIFICATIONS}}
\vspace{-5pt}

\begin{itemize}
	\item Machine Learning Specialization, Coursera (2023).
	\item Neural Networks and Deep Learning, Coursera (2022).
	% \item Quantum Computing and Implementation in Python, Qorpi Engineering Workgroup – KNTU Innovation Center (2022).
	% \item Quantum Information and Computers, Interdisciplinary Schools, Sharif University of Technology (2021).
\end{itemize}


\vspace{-0.15in}
\rule{\textwidth}{1pt}\\
\vspace{-0.15in}

{\Large \textbf{WORKSHOPS, SCHOOLS, AND CONFERENCES ATTENDED}}
\vspace{0.1in}

\begin{school}{Nov 2024}{Shahid Beheshti University, Tehran, Iran}{fMRI Image Processing With CONN Toolbox}
\vspace{-0.1in}
	\begin{itemize}
	    \item Gained hands-on experience in preprocessing, denoising, and connectivity analysis for resting-state and task-based fMRI using the CONN toolbox.
	\end{itemize}
\end{school}


% \begin{school}{Sep 2024}{Shahrood University of Technology, Semnan, Iran}{5th National Conference on Quantum Information and Computing}
% \vspace{-0.1in}
% 	\begin{itemize}
% 	    \item Attended presentations on the latest advancements in quantum information theory and quantum computing.
% 	\end{itemize}
% \end{school}


\begin{school}{Oct -- Dec 2023}{School of Biological Sciences, IPM, Tehran, Iran}{The School of Evolutionary Dynamics of Cells and Viruses}
\vspace{-0.1in}
	\begin{itemize}
	    \item Participated in lectures and discussions on evolutionary dynamics in cells and viruses.
	    \item Explored theoretical models and their biological applications.
	\end{itemize}
\end{school}


% \begin{school}{Jul 2023}{Institute for Advanced Studies in Basic Science, Zanjan, Iran}{The $28^{th}$ Special School on Topics in Physics}
% \vspace{-0.1in}
% 	\begin{itemize}
% 	    \item Engaged in advanced discussions on topics like condensed matter physics, quantum mechanics, and biosensing technologies.
% 	\end{itemize}
% \end{school}



\vspace{-0.15in}
\rule{\textwidth}{1pt}\\
\vspace{-0.15in}




% \textbf{PERSONAL INFORMATION}
% \vspace{-0.1in}

% \begin{multicols}{3}[\columnsep=0cm]
% 	\begin{itemize}
% 	  \item Gender: Female
% 	  \item Nationality: Iranian
% 	  \item Date of Birth: Aug. $10^{th}$, 2000
% 	  %\item Hobbies: Walking, Reading Books, Painting
% 	\end{itemize}
% \end{multicols}

% \vspace{-0.15in}
% \rule{\textwidth}{1pt}\\
% \vspace{-0.15in}

{\Large \textbf{REFERENCES}}

\begin{itemize}
	\item \textbf{Reza Jafari}, Professor of Physics, Department of Physics and Institute for Cognitive Science and Brian, Shahid Beheshti University, Tehran, Iran.\\
	\href{tel:(+98)2129902773}{\faIcon{phone} (+98) 21 2990 2773}
	\hspace{0.5in}
	\href{mailto:g\_jafari@sbu.ac.ir}{\faIcon{envelope} g\_jafari@sbu.ac.ir}
	\hspace{0.75in}
	\href{mailto:gjafari@gmail.com}{\faIcon{envelope} gjafari@gmail.com}
	\hspace{1.14in}
	\href{https://complexity.sbu.ac.ir/professor-reza-jafari/}{\faIcon{globe}HomePage} 
	\item \textbf{S. Ali Hosseiny Esfidvajani}, Assistant Professor, Faculty of Physics, Shahid Beheshti University, Tehran, Iran.\\
	\href{tel:(+98)2129905043}{\faIcon{phone} (+98) 21 2990 5043}
	\hspace{0.5in}
	\href{mailto:al\_hosseiny@sbu.ac.ir}{\faIcon{envelope} al\_hosseiny@sbu.ac.ir}
	\hspace{0.5in}
	\href{mailto:alihd22@gmail.com}{\faIcon{envelope} alihd22@gmail.com}
	\hspace{1.091in}
	\href{https://alihosseiny.com/}{\faIcon{globe}HomePage} 
	\item \textbf{Marzieh Farhang}, Associate Professor, Faculty of Physics, Shahid Beheshti University, Tehran, Iran.\\
	\href{tel:(+98)2129905053}{\faIcon{phone} (+98) 21 2990 5053}
	\hspace{0.5in}
	\href{mailto:m\_farhang@sbu.ac.ir}{\faIcon{envelope} m\_farhang@sbu.ac.ir}
	\hspace{0.54in}
	\href{mailto:marzieh.farhang@gmail.com}{\faIcon{envelope} marzieh.farhang@gmail.com}
	\hspace{0.5in}
	\href{https://en.sbu.ac.ir/~m_farhang/home}{\faIcon{globe}HomePage} 
\end{itemize}
\end{document}
